\documentclass[a4paper,10pt]{article}
\usepackage{etex}
\usepackage[spanish,es-noquoting]{babel}
\usepackage[utf8]{inputenx}
\usepackage{booktabs}
\usepackage{multirow}
\usepackage{amssymb}
\usepackage{graphicx}
\usepackage{listings}
\usepackage{verbatim}
\usepackage{diagrams}
\usepackage{proof}
\usepackage{amsmath}
\usepackage[all,color]{xy}
\newcommand{\here}{\checkmark}

\newenvironment{code}{\footnotesize\verbatim}{\endverbatim\normalsize}
\DeclareUnicodeCharacter{03BB}{$\lambda$}
\DeclareUnicodeCharacter{2192}{$\rightarrow$}

\usepackage{tikz}
\usetikzlibrary{arrows}

\tikzset{
  treenode/.style = {align=center, inner sep=0pt, text centered,
    font=\sffamily},
  arn_n/.style = {treenode, circle, white, font=\sffamily\bfseries, draw=black,
    fill=black, text width=1.5em},% arbre rouge noir, noeud noir
  arn_r/.style = {treenode, circle, red, draw=red, 
    text width=1.5em, very thick},% arbre rouge noir, noeud rouge
  arn_x/.style = {treenode, rectangle, draw=black,
    minimum width=0.5em, minimum height=0.5em}% arbre rouge noir, nil
}

\newcommand{\typejud}[3] {
  \ensuremath{#1 \vdash #2 : #3}
}

\newcommand{\df}[1]{\textcolor{gray}{\mathsf{#1}}}
\newcommand{\dfH}[1]{\textcolor{blue}{\mathsf{#1}}}

\title{Language Implementation: Attribute Grammars}
\author{Alejandro Gadea \and Emmanuel Gunther}

\begin{document}
\maketitle
\lstset{ language=Haskell
       , literate={λ}{{$\lambda$}}1{→}{{$\rightarrow$}}1
	   }

\section{Syntax and Semantics of Languages}

\subsection{The RepMin problem}

The ``Rep Min'' is a well known example of the power of Lazy Evaluation and inspires the way
of define semantics that exploit the Attribute Grammars.

Think about the data type of binary trees, with numbers in leaves. The problem consists in
replace each value at leaves by the least value:\\

\begin{minipage}{.37\textwidth}
\begin{tikzpicture}[level/.style={sibling distance = 3cm/#1,
  level distance = 0.7cm}] 
\node [arn_n] {}
    child{ node [arn_r] {4} }
    child{ node [arn_n] {}
            child{ node [arn_n] {} 
							child{ node [arn_r] {3}}
							child{ node [arn_r] {10}}
            }
            child{ node [arn_r] {4} }
		}
;
\end{tikzpicture}
\end{minipage}%
\begin{minipage}{.18\textwidth}
\begin{tikzpicture}
\coordinate (a) at (0,0);
\coordinate (b) at (2,0);
\draw[-angle 90, red!40!white, line width=9pt](a) to node[black]{rep\_min} (b) ;
\end{tikzpicture}
\end{minipage}
\begin{minipage}{.2\textwidth}
\begin{tikzpicture}[level/.style={sibling distance = 3cm/#1,
  level distance = 0.7cm}] 
\node [arn_n] {}
    child{ node [arn_r] {3} }
    child{ node [arn_n] {}
            child{ node [arn_n] {} 
							child{ node [arn_r] {3}}
							child{ node [arn_r] {3}}
            }
            child{ node [arn_r] {3} }
		}
;
\end{tikzpicture}
\end{minipage}

\

\

We can define a data type in Haskell for represent a binary tree of the following way:

\begin{lstlisting}
 data Tree =   Leaf Int 
	     | Bin (Tree,Tree)
\end{lstlisting}

A tree can be built from an integer (a leaf) or by others two subtrees, i.e we
can built an element of $Tree$ from an element of the disjoin union of 
$Int \sqcup (Tree\,\times\,Tree)$. To this disjoin union we can define in Haskell
for any type $a$:

\begin{lstlisting}
 type FTree a = Either Int (a,a)
\end{lstlisting}

\noindent and then we can get any element of $\mathbf{Tree}$ from some element
of $\mathbf{FTree\,Tree}$.

To the type of this function, which consist in get an element of type $a$ from one of type
$\mathbf{FTree}\,a$, explicitly we define:

\begin{lstlisting}
 type FTreeAlgebra a = FTree a -> a
\end{lstlisting}

Now, if we want to define a semantic for the binary trees in the semantic set $B$ we can
define a $\mathbf{FTreeAlgebra}\;B$. This is, give a rule in the case of we have an integer
and another in the case of have a pair in $B \times B$.

The syntactic representation of binary trees with type $Tree$ can be seen as
a trivial semantic and we can define the next algebra:

\begin{lstlisting}
  init_algebra :: FTreeAlgebra Tree
  init_algebra = either Leaf Bin
\end{lstlisting}

Thinking in categories, we can see $FTree$ as a functor which transform an object $A$ in
an object $Int \sqcup (A \times A)$ and then a FTree-algebra will be an arrow 
$\alpha\,:\,FTree\;A \rightarrow A$. If $\alpha$ is the initial FTree-algebra, then for
any other FTree-algebra $\beta\,:\,FTree\;B \rightarrow B$ must exist an unique arrow
between $\alpha$ and $\beta$, i.e, a single $f$ such that the following diagram commutes:

\begin{center}
\begin{diagram}
   FTree \ A & & \rTo^{\alpha} & & A \\
   \dTo^{FTree \ f} & & & & \dTo_{f} & \\
   FTree \ B & & \rTo^{\beta} & & B &
\end{diagram}
\end{center}

\

\

Thus, if we want to define a semantic for the binary trees we must find the single arrow
between the $init\_algebra$ (the initial FTree-algebra) and other FTree-algebra. This arrow
is called \textbf{catamorfism}:

\begin{lstlisting}
  cataTree :: FTreeAlgebra b -> Tree -> b
  cataTree beta (Leaf i)    = beta (Left i)
  cataTree beta (Bin (t1,t2)) = beta (Right (cataTree beta t1,
					     cataTree beta t2))
\end{lstlisting}

Returning to the problem that we want to solve, given a tree we want to transform it
in a tree in which the value of the leaves is now the least value.

One first solution is, first get the least value of the tree and then replace it in
all leaves. This implies walk the initial tree two times.

Now, to get the least value we define a FTree-algebra and define a function that
given an integer $n$ constructs the FTree-algebra which replace the value of leaves
for $n$:

\begin{lstlisting}
  min_alg :: FTreeAlgebra Int
  min_alg = either id (uncurry min)

  rep_min_alg :: Int -> FTreeAlgebra Tree
  rep_min_alg n = either (const $ Leaf n) Bin
\end{lstlisting}

Then, we solve the problem calling twice the function $cataTree$:

\begin{lstlisting}
  replace_min :: Tree -> Tree
  replace_min t = let n = cataTree min_alg t in
		      cataTree (rep_min_alg n) t
\end{lstlisting}

Notice that in the function $rep\_min\_alg$ we construct a FTree-algebra from an integer.
Instead of this, we can define a FTree-algebra which allow us to compute a
function that given an integer builds up a tree with all the values of the leaves
replaced by that integer:

\begin{lstlisting}
  rep_min_alg' :: FTreeAlgebra (Int -> Tree)
  rep_min_alg' = either (const Leaf) 
                        (\(lfun,rfun) -> \m -> 
			    Bin (lfun m,rfun m))
\end{lstlisting}

\noindent so, the solution to our problem can be:

\begin{lstlisting}
  replace_min' :: Tree -> Tree
  replace_min' t = (cataTree rep_min_alg' t) (cataTree min_alg t) 
\end{lstlisting}

In the previous definition we can see that the two calls of $cataTree$ are independent, thus
we could compute a single FTree-algebra that get both results at the same time. With $min\_alg$
make the least value and with $rep\_min\_alg'$ the function for builds up the tree, whereby if
we could make the product of those FTree-algebra we would get both results:

\begin{lstlisting}
  infix 9 `x`

  x :: FTreeAlgebra a -> FTreeAlgebra b -> FTreeAlgebra (a,b)
  fa `x` fb = either  (\i -> (fa $ Left i,fb $ Left i))
		      (\((a,b),(a',b')) -> 
			(fa $ Right (a,a'),fb $ Right (b,b')))
			  

  rep_min_alg'' :: FTreeAlgebra (Int,Int -> Tree)
  rep_min_alg'' = min_alg `x` rep_min_alg'
  
\end{lstlisting}

So now we can get the result to the problem making one call to $cataTree$:

\begin{lstlisting}
 replace_min'' :: Tree -> Tree
 replace_min'' t = r m
    where (m, r) = cataTree rep_min_alg'' t
\end{lstlisting}

In final step, we can see that from a value in $(Int,Int \rightarrow Tree)$ we can
obtain a value in $Int \rightarrow (Int,Tree)$. The inverse is not always true but 
is not a matter here.

We can consider the following function that given two isomorphic types
$a$ and $b$, and a $FTreeAlgebra\;a$ obtain a $FTreeAlgebra\;b$:

\begin{lstlisting}
getIsoAlg :: (a -> b) -> (b -> a) -> FTreeAlgebra a -> FTreeAlgebra b
getIsoAlg fab fba fa = either (fab . fa . Left)
                                 (fab . fa . Right . (fba *** fba))
\end{lstlisting}
\medskip

In our problem we have a FTree-algebra of type
  
\begin{lstlisting}
   (Int,Int -> Tree)
\end{lstlisting}

\noindent and we want to define a FTree-algebra of type:

\begin{lstlisting}
   Int -> (Int,Tree)
\end{lstlisting} 

We need define two functions to exchange those types:
  
  \begin{lstlisting}
  f1 :: (Int,Int -> Tree) -> (Int -> (Int,Tree))
    
  f2 :: (Int -> (Int,Tree)) -> (Int,Int -> Tree)
  \end{lstlisting}

We can construct the first of these easily, but it's not the case for the second one:

\begin{lstlisting}
f1 :: (Int,Int -> Tree) -> (Int -> (Int,Tree))
f1 (i,f) = const i &&& f

f2 :: (Int -> (Int,Tree)) -> (Int,Int -> Tree)
f2 f = (??? , snd . f)
\end{lstlisting}

We don't have too many options to define the first component of $f2\;f$. We can
return a constant integer or we can evaluate $f$ on a constant value. If we do this,
we obtain one side of the isomorphism and we can see that it is sufficient to our problem
(to more information you can ask to Magoya)

  \begin{lstlisting}
  f1 :: (Int,Int -> Tree) -> (Int -> (Int,Tree))
  f1 (i,f) = const i &&& f

  f2 :: (Int -> (Int,Tree)) -> (Int,Int -> Tree)
  f2 f = (fst (f 0) , snd . f)
  \end{lstlisting}

  Then, we define a new FTree-algebra:
  
  \begin{lstlisting}
   rep_min_alg''' :: FTreeAlgebra (Int -> (Int,Tree))
   rep_min_alg''' = getIsoAlg f1 f2 rep_min_alg''
  \end{lstlisting}

  And the solution is reached defining the next function that it works because of laziness of Haskell:
  
  \begin{lstlisting}
   replace_min''' :: Tree -> Tree
   replace_min''' t = r
      where (m, r) = (cataTree rep_min_alg''' t) m
  \end{lstlisting}

  

% de los árboles binarias con el tipo $Tree$ constituyen el álgebra inicial, que consiste en
% aplicar la función $Leaf$ si el elemento es un entero, y la función $Bin$ si 

\subsection{Repmin with Attribute Grammars}

In the previous section we saw how to solve the Repmin problem in a more efficient way, performing
a single walk to the tree thanks to the Lazy Evaluation of Haskell. For this we obtain a tuple in
which the first element was the least value of the tree and the second one was a function that
given an integer builds up a tree with all the values of leaves replaced for
that integer.

Notice that we could take the same steps for any other recursive data type, where the grammar
not necessarily has two productions. We could define a function $F$ and the appropriate catamorphism,
and then for calculate any semantics we only need to give one rule for each production of the
grammar.

With \textbf{Attribute Grammars}, we have a simple notation for define semantics, defining the
syntax of a data type and then giving the rules for each production of the grammar. A preprocessor
transform the AG notation into Haskell code that is equivalent to that obtained in the previous
section.

The Repmin problem solved with Attribute Grammars has the following possible solution:

\begin{lstlisting}
data Tree
    | Leaf      Int
    | Bin lt :: Tree
	      rt :: Tree

deriving Tree : Show
	  
attr Tree
    inh minval :: Int
    syn m      :: Int
    syn res    :: Tree
      
sem Tree
    | Leaf lhs.m   = @int
              .res = Leaf @lhs.minval
	| Bin  lhs.m   = @lt.m `min` @rt.m
              .res = Bin @lt.res @rt.res

data Root
	| Root Tree
      
      
attr Root
	syn res :: Tree
      
sem Root
	| Root tree.minval = @tree.m
\end{lstlisting}

The definition of the grammar is similar to the given definition in Haskell, just that we put names
in each parameter of each production. To calculate the semantics we define two kinds of attributes,
the synthesized ($syn$) which are those that for compute the value in a node we need to know the value
of the children and the inherited ($inh$) which are passed top to down in the tree.

In our example we need the least value of the tree to compute the result of Repmin. We define
an inherited attribute $minval$ and two synthesized attributes, $m$ to calculate the least value
of the tree and $res$ to compute the final tree with the replaced of $minval$ that is initialize
in $m$.
  
In the previous piece of code we use many abbreviations that allows to write shorter code. For example,
if a rule of a grammar production has a single argument, the name of this can be omitted and the preprocesor generates 
one with the name of the data type which has the first letter in lowercase. 
If we want define a rule for an inherited attribute which doesn't change at child node, 
it's not necessary to define that trivial rule, we can omit it and the preprocessor generates it for us.

The generated code using \textbf{Attribute Grammars} compute the attributes values by doing the
procedure that we show in the previous section. The type of the resulting algebra will be a function
where the parameters are the inherited attributes and the result will be a tuple where each element
is a synthesized attribute.
  

\section{The Lambda Calculus}

In the previous section we saw how we can define a function in a general way, that we called
catamorphism, and how this definition leads us the introduction of a special syntax
with the idea that we just need to implement the relating things of the problem.

Following, we will implement type inference for the simply typed lambda calculus, as
well as the parser and pretty printing.

\subsection{Syntax: Parser y Pretty Printing}

\subsubsection{Syntax}

$Var$ $::=$ Countable set. In general will be $a$, $b$, $c$, $x$, $y$, $z$, etc.

\begin{lstlisting}
Term ::= Var
       | λ Var . Term
       | Term Term
\end{lstlisting}

\subsubsection{Parser}

The implementation of a parser for the syntax of the lambda calculus it's rather simple
and straightforward, yet even adding the restriction that we only allow to parse closed
terms, i.e. terms in which don't occur free variables.
However, suppose we have the next term to parse,\\

$\lambda a . s$\\

clearly ``s'' is free, but would not be bad assume that in reality that term might well
have been\\

$\lambda a . a$\\

i.e. a mistake was made by writing the term. Thus, a parser a little bit
more interesting it would be one that detect that kind of errors, corrects
and gives information about such decision.\\

Now, for the implementation of the parser we use the library ``uu-parsinglib''
which provides exactly a error correction. For example, being ``pa'' the parser
of the letter ``a'', trying to parse ``b'' will produce an ``a'' where what 
happened was the replacement of ``b'' for ``a'':\\

\begin{code}
 >>> run pa  "b"
     Result: "a"
     Correcting steps: 
       Deleted   'b' at position LineColPos 0 0 0 expecting 'a'
       Inserted  'a' at position LineColPos 0 1 1 expecting 'a'
\end{code}

Some to highlight is that the error correction that we pretend for our parser
of terms is almost free. In a preliminary summary, for parse the body of an 
abstraction we just need to have a list with the introduced variables till that
moment and generate parsers for each name of variable on the list.\\

Before introduce the parsers concerning each construction of the grammar, 
we define some general parsers that will be useful.

\begin{lstlisting}
parseTermSym :: Parser String -> [String] -> Parser String
\end{lstlisting}
We generate a parser from a default parser and a list of strings.

\begin{lstlisting}
parseXSym :: Parser String
\end{lstlisting} con X $\in$ {Lam,Dot,App}\\
Parsers for the symbols $\lambda$, $.$ y $@$. Where further we can
parser ``\textbackslash \textbackslash'' instead of $\lambda$ and $->$ or $\rightarrow$
instead of ``$.$'' .

\begin{lstlisting}
parseVar :: Parser Var
\end{lstlisting} Parser for variables.\\

Now, given the list of variable to parse, say $vars$, parse an variable identifier
will be generate parsers for each variable in the list and fail as default parser:

\begin{lstlisting}
parseId :: [Var] -> Parser Term
parseId vars = Id <$> parseTermSym pFail vars
\end{lstlisting}

Something interesting to highlight is that in the generation of the parsers of
variables is contained the action of correct an occurrence of free variable.\\

For the case of the abstraction, leaving aside the parse of the respective symbols,
we going to parse a variable and we could say that we needed for two things;
(1) the constructor of the abstraction, (2) the list of possibles variables to parse.
This bring a problem, since when we parse a variable this is encapsulated inside
of the computation ``Parser Var'', by (2) we can think that the type of the parser
that parse the body of the abstraction is $Var -> Parser Term$, because take the
variable, adds to the list of variables and then parse the term.
If now we pay attention we need a function that combines, parse the variable with
parse a term adding that variable, i.e. a function with type:

\begin{lstlisting}
Parser Var -> (Var -> Parser Term) -> Parser Term
\end{lstlisting}

but this is just the type of the bind ($>>=$), $m \ a \rightarrow (a \rightarrow m \ b) \rightarrow m \ b$. This force us to use monads, for which we need to use the addLength function. We are
not sure if exist a way of solve this without monads, with the purpose of avoid the
use of addLength and of course... monads. Then, the final definition is,

\begin{lstlisting}
parseAbs :: [Var] -> Parser Term
parseAbs vars = 
       addLength 1 $
       join $ uncurry (<$>) 
           <$> 
       (Abs &&& parseTerm . (:vars))
       	   <$ 
       parseLamSym <*> parseVar <* parseDotSym
\end{lstlisting}

Concluding, for the case of the application will be parse variable identifier or
abstraction separated by ``$@$'', for this we define a particular parser that
contain these parsers,

\begin{lstlisting}
parseTerm' :: [Var] -> Parser Term
parseTerm' vars =  parseId vars
               <|> parseAbs vars
               <|> pParens (parseTerm vars)

parseApp :: [Var] -> Parser Term
parseApp vars = (App <$ parseAppSym) `pChainl` (parseTerm' vars)
\end{lstlisting}

then, parse a term will be simply the parser defined for the application,

\begin{lstlisting}
parseTerm :: [Var] -> Parser Term
parseTerm vars = parseApp vars
\end{lstlisting}

In conclusion, we have a parser for the previous grammar which also corrects some
mistakes, as we mentioned at the start. Something important to comment is that the
correction of errors may have some that other unwanted behaviour due the ``simplistic''
implementation in terms of the use of the power of correction of the library. An
example of this may be,

\begin{verbatim}
>>> parserTerm "\\a -> b@a"
(λ a → a
, [-- Deleted   'b' at position LineColPos 0 6 6 expecting Whitespace
  ,-- Deleted   '@' at position LineColPos 0 7 7 expecting "a"
  ]
)
\end{verbatim}

where most probably the wanted correction it would have been replace ``b'' for
``a'', and generate the term $\lambda$ $a$ $\rightarrow$ $a@a$.

\subsubsection{Pretty Printing}

The implementation of the pretty printing interests us as a first little step to
the definition of the semantic function for the data type $Term$ using AG. In
summarize, the semantics that we are thinking transform a $Term$, i.e. a term
of the lambda calculus, in his representation in $String$.\\

Despite of having few constructors, a term of the lambda calculus can be complex
to read and an ``organization'' of the subterms when it comes to writing can be
very helpful for the interpretation. For example, the term 
$(\lambda x \rightarrow x@x)@(\lambda x \rightarrow x@x)$
not bring much trouble, but however if we have,
$(\lambda x \rightarrow
	\lambda y \rightarrow
		\lambda f \rightarrow f @ (x @ (\lambda w \rightarrow f @ w)) @ (y @ f @ x))$
$@$
$(\lambda x \rightarrow 
	\lambda y \rightarrow \lambda f \rightarrow 
	\lambda g \rightarrow \lambda h \rightarrow h @ (f @ x) @ (g @ y))$
$@$ $\lambda x \rightarrow \lambda f \rightarrow 
		\lambda g \rightarrow f @ g @ (x @ g)$
it must be very difficult to read. Thus, a good presentation of the term it always
helpful. The use of the ``uulib'' library, in particular ``UU.PPrint'', gives us
a way of implement a pretty printing in a simple way.\\

As we saw for the case of Repmin, we must define attributes and give rules for
each constructor, this time for $Term$, which express how to transform any
in his representation in $String$.\\

We want an synthesised attribute, that we called ``pprint'', that is the result
of transform a $Term$ in a $String$ and an inherited attribute that we need to
know when enclose a term between brackets or not.

\begin{lstlisting}
attr Term 
    syn pprint :: {Doc}
    inh paren  :: {ParenInfo}
\end{lstlisting}

\noindent On the other hand the semantic is defined as,

\begin{lstlisting}    
sem Term
\end{lstlisting}
For the variable identifier will be simply print the name of the variable, remembering
that these were represented by $String$s
\begin{lstlisting}    
    | Id  lhs.pprint  = text @ident
\end{lstlisting}

In the case of the application, we began explaining that print an application will
be simply print is left side (lt) and is right side (rt) but having some considerations;
as for the brackets, because the application is left associative if we have, for example
the application of $x@y$ and $z@w$ we will need to place parenthesis only in the right side
($rt.paren = Paren$) for print the correct application $x@y@(z@w)$. But also, there is
a particular case where the left side need be enclosed in parentheses, for example
applying $\lambda x \rightarrow x$ and $z@w$. That is the reason of only enclose between
parenthesis in the left side if this is an abstraction ($lt.paren = ParenAbs$).
Notice that without the parenthesis in the example applications they would be $x@y@z@w$ and
$\lambda x \rightarrow x@z@w$ which not are the correct representations of the original
terms.\\

Now, leaving a side the parentheses, for print an application the idea would be try that
both, the left and right side, stay at the same level and in the case that this is not
possible obtain the following,

\begin{center}
$lt$\\
$@$\\
$rt$
\end{center}

i.e, divide the subterms in two levels intercalating the application operator. With the
possibility of put the right side at the same level of the operator.

\begin{lstlisting}    
    | App lhs.pprint  = (putParen @lhs.paren) 
                (group $ @lt.pprint <> line <> 
                  (group $ text "@" <> line <> @rt.pprint)
                )
          rt.paren    = Paren
          lt.paren    = ParenAbs
\end{lstlisting}    

Finishing, we need to define the semantics for the case of the abstraction constructor.
For this case the idea will be, if we can write all the abstraction in the same level
we do it, when we can't we will print the body ($term$) in other level with some
indented (3 characters) for help to identify the body of the abstraction when this
is being applied to other term. On the other hand, the body of the abstraction never
needs to be between parentheses ($term.paren = NoParen$).

\begin{lstlisting}
    | Abs lhs.pprint  = (putAbsParen @lhs.paren)
                (text "λ " <> text @var <> text " →" <> 
                    group (nest 3 $ line <> @term.pprint)
                )
          term.paren  = NoParen
\end{lstlisting}

Printing the example of the beginning we can note that is much easier to read. In
particular contains many of the print cases that the semantics we give implements.

\begin{verbatim}
>>> App (App t2 t3) t1
(λ x →
   λ y →
      λ f → f @ (x @ (λ w → f @ w)) @ (y @ f @ x))
@
(λ x →
   λ y → λ f → λ g → λ h → h @ (f @ x) @ (g @ y))
@ (λ x → λ f → λ g → f @ g @ (x @ g))
\end{verbatim}

\subsection{Inferidor de tipos}


 The main purpose of this work is to define type inference for the simply typed lambda calculus. Remind the typing rules:
 
 \begin{align*}
 \infer[^{\mathtt{T-VAR}}]
       {\typejud{\pi,(v:\theta)}{v}{\theta}}
       {}
 \end{align*}
  \begin{align*}
  \infer[^{\mathtt{T-APP}}]
       {\typejud{\pi}{t_1\;t_2}{\theta_2}}
       {\typejud{\pi}{t_1}{\theta_1\rightarrow\theta_2} &
        \typejud{\pi}{t_2}{\theta_1}
       }
  \end{align*}
  \begin{align*}
  \infer[^{\mathtt{T-ABS}}]
       {\typejud{\pi}{\lambda\,v.t}{\theta_1 \rightarrow \theta_2}}
       {\typejud{\pi,(v:\theta_1)}{t}{\theta_2}
       }
  \end{align*}
 
 In a context $\pi$ a term $t$ will have type $\theta$ if exists a derivation of
 $\typejud{\pi}{t}{\theta}$ using the previous rules.
 
 To infer the type of a term we can assume that it has type $\theta$ and then perform substitutions
 to this type until obtain a valid type judgement. Supose we have the next term:
 
 \begin{align*}
    \lambda\,x.x
 \end{align*}

 We assume that it has type $\theta$, but as this is an abstraction, the only rule to construct a valid
 judgement is $\mathtt{T-ABS}$ where the type has to be in the form $\theta_1 \rightarrow \theta_2$,
 so we can replace $\theta\,:=\,\theta_1 \rightarrow \theta_2$ and then we try to construct type judgement
 $\typejud{\pi,(x:\theta_1)}{x}{\theta_2}$. As the term now is a variable, the only possible rule
 is $mathtt{T-VAR}$ and we need to replace $\theta_1$ for $\theta_2$. Finally the inferred type is:
 
  \begin{align*}
    {\theta_2 \rightarrow \theta_2}
  \end{align*} 
 
 \medskip
 
  With this idea in mind we'll implement type inference using UUAG library. First at all we define the grammar:
  
  \begin{lstlisting}
   data Term
	| Id  ident :: {Var}
	| App lt,rt :: Term
	| Abs var   :: {Var}
	      term  :: Term
  \end{lstlisting}

  Where $Var$ is defined as $String$.
  
  Types are defined in normal Haskell code with the next grammar:
  
  \begin{lstlisting}
   data Type = AtomType TVar | FunType Type Type
  \end{lstlisting}

  \indent and $TVar$ is implemented with integer numbers.
  \medskip
  
  We define the attributes of the grammar for lambda terms:
  
  \begin{lstlisting}
   attr Term 
      inh ctx       :: Ctx
      inh termType  :: Type
      chn maxTVar   :: TVar
      syn tsubst    :: TSubst
  \end{lstlisting}

  The result of type inference will be a substitution to apply on the initial type, and we implement it with a
  syntesized attribute $\mathbf{tsubst}$. As we saw in the previous example, given a term we assume that it has
  a type, this assumed type is implemented wit an inherited attribute $\mathbf{termType}$. We'll have a context
  for assign types to variables too, so we define the attribute $\mathbf{ctx}$ and because is necessary 
  to generate fresh type variables, we define the chained attribute $\mathbf{maxTVar}$, so when we need
  to get a fresh type, it will be the following type variable to the $maxTVar$. This attribute is chained
  because is necessary to pass the value in a transversal way.
  
  To see how the implemented algorithm works consider the next term:
  
 \begin{align*}
    \lambda\,f.\lambda\,x.f\;(f\;x)
 \end{align*}

 \noindent which is defined in Haskell:
 \begin{lstlisting}
  Abs "f" (Abs "x" (App (Id "f") (App (Id "f") (Id "x"))))
 \end{lstlisting}
 
 \noindent or in a tree way:
 
  \begin{center}
  \begin{diagram}[h=2em]
	  & Abs \\
	  & \dTo_{f}\\
	  & Abs \\
	  & \dTo_{x}\\
	  & App\\
	  \ldTo(1,2) & & \rdTo(1,2)\\
	  f & & App\\
      & \ldTo(1,2) & & \rdTo(1,2)\\
	  & f & & x
  \end{diagram}
  \end{center}

  For clarity in explanation of the example we omit some details, for example the value of attribute $maxTVar$ needed to 
  generate fresh variables won't be considered.
  \medskip
  
  The first node is an abstraction. We have to define the value of the inherited attributes on child and
  to calculate the syntesized attribute in function of it.
  

  \xymatrix @C=0.1pc @R=0.3pc{
  & & & & & \ar@{-->}@/_-0.3pc/[dddd] & \ar[dddd]^{f} Abs & \ar@{<--}@/_0.3pc/[dddd] & & & & & & & & &\\
  & & & \dfH{\pi = f:\theta_1} & & &  & & & & & & & & & &\\
  & & & \dfH{t = \theta_2} & & &  & & \dfH{TS = \{\theta:=\theta_1 \rightarrow \theta_2\} \bigcup \,\{?\}}& & & & & & & &\\
  & & & \dfH{UV = \{\theta,\theta_1,\theta_2\}} & & &  & & & & & & & & & &\\
  & & & & &  &  Abs &  & & & & & & & & &\\\\
  }
  
  As we assume for this term the type $\theta$, we must to replace by $\theta_1 \rightarrow \theta_2$ and
  we define the values of the inherited attributes: the context will contain the pair $f:\theta_1$ and
  the asummed type on child will be $\theta_2$. The substitution will contain $\theta : \theta_1 \rightarrow \theta_2$
  and the substitution obtained from child. In the graphics we'll complete the syntesized attribute as we calculate 
  the attributes in subterms.
  \medskip
  
  In next step, the node is an abstraction too, so we perform the same procedure:
   
    \xymatrix @C=0.1pc @R=0.3pc{
    & & & & & \ar@{-->}@/_-0.3pc/[dddd] & \ar[dddd]^{f} Abs & \ar@{<--}@/_0.3pc/[dddd] & & & & & & & & &\\
    & & & \df{\pi = f:\theta_1} & & &  & & \dfH{TS = \{\theta:=\theta_1 \rightarrow \theta_3 \rightarrow \theta_4,} & & & & & & & &\\
    & & & \df{t = \theta_2} & & &  & & \;\;\;\;\;\;\;\;\;\;\dfH{\theta_2 := \theta_3 \rightarrow \theta_4\} \bigcup\,\{?\}} & & & & & & & &\\
    & & & \df{UV = \{\theta,\theta_1,\theta_2\}} & & &  & & & & & & & & & &\\
    & & & & & \ar@{-->}@/_-0.3pc/[dddd] & \ar[dddd]^{x} Abs & \ar@{<--}@/_0.3pc/[dddd] & & & & & & & & &\\
    & & & \dfH{\pi = f:\theta_1,\,x:\theta_3} & & &  & & & & & & & & & &\\
    & & & \dfH{t = \theta_4} & & &  & & \dfH{TS = \{\theta_2:\theta_3 \rightarrow \theta_4\} \bigcup\,\{?\}} & & & & & & & &\\
    & & & \dfH{UV = } & & &  & & & & & & & & & &\\
    & & & & &  & App  & & & & & & & & & &\\\\
    }

  \noindent $\theta_2$ will be replaced by $\theta_3 \rightarrow \theta_4$, we add to context $x:\theta_3$ and
  the type asummed to the subterm is $\theta_4$. We update the substitution on initial node adding the obtained
  in this subterm, and we perform the replacement $\theta_2 := \theta_3 \rightarrow \theta_4$ on it, so
  we can ensure that the variables occurring on the left side of substitution don't occur on the right side. We
  perform unification of types when necessary.
  \medskip
  
  Now we analize the next node which is an application. According to typing rule, if the term type is $\theta_4$,
  the left subterm has to have a type $\theta_5 \rightarrow \theta_4$ and the right subterm $\theta_5$.
  We update the values:
  
  \xymatrix @C=0.1pc @R=0.3pc{
    & & & & & \ar@{-->}@/_-0.3pc/[dddd] & \ar[dddd]^{f} Abs & \ar@{<--}@/_0.3pc/[dddd] & & & & & & & & &\\
    & & & \df{\pi = f:\theta_1} & & &  & & \df{TS = \{\theta:=\theta_1 \rightarrow \theta_3 \rightarrow \theta_4,} & & & & & & & &\\
    & & & \df{t = \theta_2} & & &  & & \;\;\;\;\;\;\;\;\;\;\df{\theta_2 := \theta_3 \rightarrow \theta_4\} \bigcup\,\{?\}} & & & & & & & &\\
    & & & \df{UV = \{\theta,\theta_1,\theta_2\}} & & &  & & & & & & & & & &\\
    & & & & & \ar@{-->}@/_-0.3pc/[dddd] & \ar[dddd]^{x} Abs & \ar@{<--}@/_0.3pc/[dddd] & & & & & & & & &\\
    & & & \df{\pi = f:\theta_1,\,x:\theta_3} & & &  & & & & & & & & & &\\
    & & & \df{t = \theta_4} & & &  & & \df{TS = \{\theta_2:\theta_3 \rightarrow \theta_4\} \bigcup\,\{?\}} & & & & & & & &\\
    & & & \df{UV = } & & &  & & & & & & & & & &\\
  & & & & & \ar@{-->}@/_-0.3pc/[dddddlll]!<7ex,0ex> \ar@{-->}@/_0.7pc/[dddddrrr]!<-2ex,0ex> &
											  \ar[dddddlll] App \ar[dddddrr] & 
		    \ar@{<--}@/_-0.6pc/[dddddllll]!<2ex,0ex>
		    \ar@{<--}@/_0.7pc/[dddddr]!<15ex,-10ex> & & & & & & & & &\\
  & & & \dfH{\pi} & & & \dfH{\pi}  & & & & & & & & & &\\
  & & & \dfH{t = \theta_5 \rightarrow \theta_4} & & & \dfH{t = \theta_5} & & \dfH{TS = ?} & & & & & & & &\\
  & & & \dfH{UV = } & & &   & & & & & & & & & &\\
  & & & & & &   & & & & & & & & & &\\
  & & & f & & & & & App & & & & & & & &\\\\
  }
  
  The substitution on an application will be the union of the substitutions in subterms, so we don't complete this
  yet.
  \medskip

  Now we have two terms to analize. The left is the identifier with variable $y$ and asummed type is $\theta_5 \rightarrow \theta_4$
  but in the context we have $f:\theta_1$. So, we need to add $\theta_1 := \theta_5 \rightarrow \theta_4$ to the substitution.
  
  The right term is another application with type $\theta_5$, so we realize the same procedure:
  
  \begin{center}
   gráfico
  \end{center}
   
  Ahora tenemos de nuevo en el lado izquierdo el identificador $f$ y el tipo asumido es $\theta_6 \rightarrow \theta_5$,
  y como en el contexto tenemos $f:\theta_1$ agregamos a la substitución $\theta_1 := \theta_6 \rightarrow \theta_5$.
  
  Al momento de concatenar las substituciones vemos que tenemos 
  
  
  \section{Conclusions}
  
  En la primera sección de este trabajo explicamos el ejemplo ``RepMin'' el cual es introducido en varios
  trabajos sobre Attribute Grammars como en (AFP3) o (tesis del chabón). Una manera de ver la sintaxis y semántica
  de lenguajes es mediante álgebras iniciales (initial algebra semantics), y nos encontramos estudiando este tema
  utilizando teoría de categorías
  utilizando teoría de categorías y es un área que nos encontramos 
  estudiando en nuestro doctorado por lo cual nos pareció interesante introducirlo. Todavía 
  
  
  
   
\end{document}




